\documentclass[../main.tex]{subfiles}
\begin{document}
\chapter{GigaScience}
\section{La rivista}

L'articolo di interesse in questa tesi è stato pubblicato su \href{https://academic.oup.com/gigascience}{GigaScience}\footnote{URL: \texttt{https://academic.oup.com/gigascience}}, una rivista accademica a revisione paritaria attualmente pubblicata dalla Oxford University Press.\cite{gigaScienceOxford}
Questa rivista è stata fondata nel 2012 dal \acrfull{bgi} per raccogliere insiemi di dati e lavori di ricerca nelle scienze biomediche e permettere accesso libero agli articoli pubblicati sul suo sito.\cite{gigascienceIntro}

La rivista è estratta e indicizzata nei principali database bibliografici, come PubMed, Scopus, Web of Science e Google Scholar.

\subsection{Origine}

Poiché la comunità scientifica è sempre più impegnata in ricerche su larga scala e ad alta intensità di dati, i modelli tradizionali di pubblicazione accademica faticano a soddisfare il volume, la complessità e le esigenze di accessibilità di tali lavori.
Per questo motivo è stata fondata GigaScience, una nuova piattaforma editoriale progettata specificamente per supportare la produzione, la condivisione e il riutilizzo di grandi insiemi di dati biologici.

Fondata con l'obiettivo di promuovere una maggiore collaborazione scientifica internazionale, GigaScience integra la tradizionale pubblicazione di articoli con un'infrastruttura completa di hosting dei dati, GigaDB.\cite{gigadb}
A ogni set di dati associato a una pubblicazione viene assegnato un \acrfull{doi}, che consente una citazione indipendente dei dati e ne garantisce la permanenza e la visibilità.
Questo modello non solo accelera la diffusione della ricerca, ma riconosce anche formalmente il valore della generazione dei dati come contributo scientifico a sé stante.

\subsection{Portata}

Il modello innovativo di questa rivista, che combina i principi dell’open access con un’infrastruttura pensata per la gestione dei big data, la rende una fonte rilevante e affidabile per la ricerca in discipline ad alta intensità di dati.

Oltre alla genomica e alla bioinformatica, GigaScience accoglie contributi da tutte le scienze della vita che si basano su grandi insiemi di dati condivisibili, dalla neuroimmagine agli studi di popolazione.
Il suo team editoriale lavora direttamente con gli autori per affrontare le sfide logistiche dell'hosting dei dati, in particolare nei campi in cui gli archivi standard sono carenti o inesistenti.

\subsection{Modello di revisione}

I lavori che comprendono grandi quantità di dati tipicamente sono svolti da un team formato da più autori con aree di competenza diverse, il che rende difficile trovare revisori esperti in tutti gli aspetti del lavoro.
Per questo motivo, GigaScience ha sviluppato un modello di revisione paritaria  che consente una revisione più rapida e mirata dell'articolo e dei dati associati.

La revisione paritaria è suddivisa in base alle aree di competenza dei revisori (ad esempio produzione di dati, analisi computazionale, interpretazione biologica, etc), consentendo una valutazione più mirata e tempi più rapidi.
Per rendere il processo di revisione più trasparente, su GigaScience i nomi dei revisori e i commenti sono pubblicati per impostazione predefinita, a meno che non si scelga di escluderli, aggiungendo un livello di responsabilità e apertura al processo di revisione.

\section{L'articolo}

La presente tesi fa riferimento a delle immagini di un articolo pubblicato su GigaScience nell’ambito di un più ampio lavoro di analisi tramite sistemi a spettroscopia nanometrica.\cite{ssnombacter_data}

\subsection{Premessa}

I batteri patogeni, inclusi quelli del gruppo ESKAPE (come \textit{Enterococcus faecium}, \textit{Staphylococcus aureus}, \textit{Klebsiella pneumoniae}, \textit{Acinetobacter baumannii}, \textit{Pseudomonas aeruginosa} e \textit{Enterobacter spp.}), sono una delle principali cause di infezioni nosocomiali e sono spesso resistenti a più classi di antibiotici, provocando milioni di morti ogni anno.\cite{tacconelli_2018,rice_2010,pendleton_2013}

La microscopia gioca un ruolo cruciale nello studio di questi batteri, permettendo una caratterizzazione dettagliata delle strutture batteriche, tuttavia tecniche di microscopia convenzionali, come la microscopia confocale a scansione laser (\acrshort{clsm}), hanno dei limiti di risoluzione dovuti al fenomeno di diffrazione della luce, limitando la risoluzione laterale fino a ~200nm.
Di conseguenza, non è ancora stata raggiunta una comprensione esatta delle strutture e dei processi fondamentali dei batteri ed occorrono analisi a risoluzioni più elevate per studiare gli aspetti che non sono ancora ben compresi.

Alcune tecniche di nanoscopia ottica, come la nanoscopia \acrshort{sted}, \acrshort{palm} o \acrshort{storm}, pur superando questa barriera, offrendo risoluzioni nell'ordine delle decine di nanometri, hanno una scarsa sensibilità chimica e richiedono delle sonde fluorescenti molto specifiche, che ne limitano le applicazioni.
Queste considerazioni, unite a recenti studi su possibili interazioni tra gli agenti di contrasto e i campioni biologici\cite{cosentino_2019}, hanno portato ad un aumento di interesse per sistemi di imaging ottico che non necessitano agenti di contrasto.

\subsection{Imaging senza contrasto}

Le tecniche di microscopia ottica senza contrasto, sia quelle a campo lontano sia quelle a campo prossimo, come la microscopia \acrshort{ssnom}, hanno il potenziale per far avanzare la nostra  conoscenza delle caratteristiche strutturali, chimiche e ottiche di campioni biologici e di biomateriali avanzati. Queste sono, tuttavia, delle tecniche relativamente recenti e il loro uso è limitato a causa dei costi associati e delle competenze richieste.
Anche l'accesso ai set di dati raccolti con queste tecniche è limitato, portando a ritardi nel loro trasferimento ad applicazioni fuori dall'interesse scientifico del gruppo che li ha sviluppati. Oltretutto, i metodi moderni per l'analisi automatica delle immagini biologiche hanno avuto sviluppi insignificanti per queste modalità emergenti senza contrasto.

\end{document}
