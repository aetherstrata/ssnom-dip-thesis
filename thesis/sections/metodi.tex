\documentclass[../main.tex]{subfiles}
\begin{document}
	
	\chapter{Metodi di elaborazione delle immagini}
	
	Nel lavoro svolto in questa tesi sono state prese in esame le immagini di topografia \acrshort{afm} e le ampiezze della $2^a$ e $3^a$ armonica \acrshort{ssnom} per valutare l'efficacia di diverse tecniche di riduzione del rumore su questo tipo di immagini. Il rumore dipende anche dal tempo di integrazione per pixel del microscopio: impostare un periodo di tempo più alto permette di avere immagini con meno rumore a discapito di un tempo di scansione più lungo. Usare delle tecniche di riduzione del rumore su immagini ottenute con un tempo di integrazione basso, come nel caso delle immagini del dataset SSNOMBACTER, permette di aumentare la qualità delle immagini senza dover aumentare il tempo richiesto per l'acquisizione delle immagini.\cite{baiz_2025} 
	
	\begin{figure}[ht]
		\centering
		\begin{subfigure}{0.32\linewidth}
			\centering
			\includegraphics[keepaspectratio, width=\linewidth]{images/kp_o0a.png}
			\caption{O0A}
		\end{subfigure}
		\begin{subfigure}{0.32\linewidth}
			\centering
			\includegraphics[keepaspectratio, width=\linewidth]{images/kp_o1a.png}
			\caption{O1A}
		\end{subfigure}
		\begin{subfigure}{0.32\linewidth}
			\centering
			\includegraphics[keepaspectratio, width=\linewidth]{images/kp_o2a.png}
			\caption{O2A}
		\end{subfigure}\\[4pt]
		\begin{subfigure}{0.32\linewidth}
			\centering
			\includegraphics[keepaspectratio, width=\linewidth]{images/kp_o3a.png}
			\caption{O3A}
		\end{subfigure}
		\begin{subfigure}{0.32\linewidth}
			\centering
			\includegraphics[keepaspectratio, width=\linewidth]{images/kp_o4a.png}
			\caption{O4A}
		\end{subfigure}
		\begin{subfigure}{0.32\linewidth}
			\centering
			\includegraphics[keepaspectratio, width=\linewidth]{images/kp_o5a.png}
			\caption{O5A}
		\end{subfigure}
		\caption[Immagini in ampiezza SNOM di esemplari di \textit{Klebsiella pneumoniae}]{
			Immagini in ampiezza \acrshort{snom} di esemplari di \textit{Klebsiella pneumoniae} \cite{ssnombacter_data}}
	\end{figure}
	
	Nelle immagini \acrshort{ssnom}, oltre a una maggiore rilevanza dei dettagli di superficie, sono presenti diversi tipi di rumore  che invece sono assenti o molto ridotti nelle immagini della topografia AFM.
	
	\section{Tecniche di riduzione del rumore}
	
	\subsection{Filtro guassiano}
	
	Il \textbf{filtro gaussiano} è uno dei filtri più usati nell'elaborazione delle immagini come  tecnica di sfocatura, di riduzione del rumore e di eliminazione dei dettagli. Il filtro gaussiano è un filtro lineare passa-basso efficace in particolare per rimuovere il rumore soggetto a una distribuzione normale
	
	\begin{equation}
		G(x) = e^{-\frac{x^2}{2\sigma^2}}
	\end{equation}
	
	Il parametro $\sigma$ E' la varianza. Nell'elaborazione numerica delle immagini, si usa una funzione gaussiana discreta con media nulla  \cite{ito_2000}
	
	\begin{equation}
		G[i,j] = e^{-\frac{i^2+j^2}{2\sigma^2}}
	\end{equation}
	
	dove $i$ e $j$ individuano i pixel dell'immagine.
	
	La funzione gaussiana possiede alcune importanti proprietà: \cite{wang_2014,getreuer_2013}
	
	\begin{enumerate}
		\itemsep0em
		\item La funzione gaussiana bidimensionale ha simmetria rotazionale. In generale, non si conosce la direzione dei bordi nelle immagini e non si può prestabilire se applicare più smussamento in una direzione rispetto all'altra prima di filtrare. Avere una simmetria rotazionale significa che il filtro gaussiano non seleziona nessuna direzione nell'elaborazione delle immagini.
		\item La funzione gaussiana è una funzione a valore singolo. Il filtro gaussiano effettua una media ponderata dell'intorno dei pixel per sostituire il valore del pixel centrale. Il peso è monotono decrescente con la distanza dal punto centrale quindi avrà scarso effetto sui pixel che sono lontani dal centro.
		\item La larghezza del filtro gaussiano è caratterizzata dal parametro $\sigma$. Più grande è la varianza, maggiore sarà la banda di frequenza del filtro gaussiano. \\[-10pt]
		
		\begin{figure}[ht]
			\centering
			\begin{subfigure}{0.4\linewidth}
				\includegraphics[keepaspectratio, width=\linewidth]{images/ec_o2a.png}
				\caption{Originale}
			\end{subfigure}
			\hspace{10pt}
			\begin{subfigure}{0.4\linewidth}
				\includegraphics[keepaspectratio, width=\linewidth]{images/gauss_05.png}
				\caption{$\sigma = 0.5$}
			\end{subfigure}\\[4pt]
			\begin{subfigure}{0.4\linewidth}
				\includegraphics[keepaspectratio, width=\linewidth]{images/gauss_1.png}
				\caption{$\sigma = 1$}
			\end{subfigure}
			\hspace{10pt}
			\begin{subfigure}{0.4\linewidth}
				\includegraphics[keepaspectratio, width=\linewidth]{images/gauss_2.png}
				\caption{$\sigma = 2$}
			\end{subfigure}
			\caption{Effetto del parametro $\sigma$ sul filtraggio}
		\end{figure}
		
		\item La funzione gaussiana è separabile in due funzioni monodimensionali. Questa proprietà è importante per la complessità computazionale di questo filtro, in quanto si può applicare una funzione gaussiana 1D in una direzione e poi nell'altra, riducendo la complessità da $O(N^2)$ a $O(2N)$.
		\begin{equation}
			e^{-\frac{x^2+y^2}{2\sigma^2}}=e^{-\frac{x^2}{2\sigma^2}}e^{-\frac{y^2}{2\sigma^2}}
		\end{equation}
		\item La trasformata di Fourier della funzione gaussiana è una funzione gaussiana. 
		\begin{equation}
			\hat{G}[u,v] = \hat{G_1}[u]\cdot\hat{G_1}[v] = \sum_{n=0}^{N-1}G[n]e^{-2\pi i\frac{nu}{N}}\cdot\sum_{m=0}^{M-1}G[m]e^{-2\pi i\frac{mv}{M}}
		\end{equation}
	\end{enumerate}
	
	\subsection{Filtro rettangolare}
	
	Il \textbf{filtro rettangolare} è un filtro lineare costante nel dominio spaziale in cui ogni pixel dell'immagine viene sostituito con la media aritmetica dei valori dei pixel nel suo intorno di raggio $r$. Utilizzando il \textit{Teorema di Lindeberg-Lévy}, è possibile dimostrare che una ripetuta applicazione di questo filtro può essere usata per approssimare un filtro gaussiano\cite{getreuer_2013}.
	
	\begin{equation}
		\operatorname{box}[i,j] = \frac{1}{(2r+1)^2}\sum_{n=-r}^{r}\sum_{m=-r}^{r} I(x+n, y+m)
	\end{equation}
	\\[-10pt]
	Utilizzando dei pesi uguali per tutti gli elementi dell'intorno, questo filtro può essere implementato utilizzando un algoritmo di accumulo molto più semplice, che è  più veloce rispetto all'utilizzo di un algoritmo di finestra mobile, passando da una complessità temporale di $O(m^2n^2)$ ad una di $O(m^2)$, dove $m$ è la grandezza dell'immagine e $n$ è la grandezza della finestra.\cite{jarosz_2001}
	
	Questo filtro, a differenza del filtro gaussiano, non può essere utilizzato efficacemente nel dominio delle frequenze in quanto si trasforma in un seno cardinale ed ha supporto infinito. Dovendo essere limitato in banda per essere utilizzato, può provocare degli artefatti di \textit{ringing} nelle regioni in cui il valore del segnale cambia repentinamente. Questo comportamento prende il nome di \textit{fenomeno di Gibbs} \cite{carslaw_1925}.\medskip
	
	\begin{figure}[ht]
		\centering
		\begin{subfigure}{0.4\linewidth}
			\centering
			\includegraphics[keepaspectratio, width=\linewidth]{images/box_spatial.png}
			\caption{Filtro}
		\end{subfigure}
		\hspace{20pt}
		\begin{subfigure}{0.4\linewidth}
			\centering
			\includegraphics[keepaspectratio, width=\linewidth]{images/box_spectrum.png}
			\caption{Spettro}
		\end{subfigure}
		\caption{Esempio di filtro rettangolare}
	\end{figure}
	
	\begin{equation}
		\begin{aligned}
			\operatorname{box}(u,v) \quad\overset{\mathcal{F}}{\longrightarrow}\quad \hat{B}(x,y) &=  \frac{1}{a^2}\int_{-a/2}^{a/2}e^{2\pi iux}\,du\int_{-a/2}^{a/2}e^{2\pi ivy}\,dv =\\
			\boxed{a = 2r + 1} \qquad\quad &= \operatorname{sinc}(ax)\cdot\operatorname{sinc}(ay)
		\end{aligned}
	\end{equation}
	
	\begin{figure}[ht]
		\centering
		\begin{subfigure}{0.4\linewidth}
			\centering
			\includegraphics[keepaspectratio, width=\linewidth]{images/ringing_orignal.png}
			\caption{Immagine originale}
		\end{subfigure}
		\hspace{20pt}
		\begin{subfigure}{0.4\linewidth}
			\centering
			\includegraphics[keepaspectratio, width=\linewidth]{images/ringing_artifact.png}
			\caption{Immagine con artefatti}
		\end{subfigure}
		\caption[Artefatti di un filtro passa-basso ideale]{
			Artefatti di un filtro passa-basso \cite{ringingImage}}
	\end{figure}
	
	\subsection{Filtro mediana}
	
	Il \textbf{filtro mediana} è un filtro simile al filtro box, con la differenza che ogni pixel dell'immagine è sostituito dalla mediana dei valori dei pixel nel suo intorno invece che la media, quindi non è un filtro lineare. Questo filtro è molto usato nell'elaborazione delle immagini perché, sotto alcune condizioni, permette di rimuovere il rumore senza modificare i dettagli dell'immagine \cite{rezaee_2021}.
	
	Se il rumore gaussiano additivo è basso, il filtro mediano è migliore del filtro gaussiano per rimuovere il rumore preservando i bordi per una determinata dimensione della finestra \cite{castro_2009}. Inoltre, questo filtro è particolarmente efficace su rumori di tipo impulsivo, come il rumore \textit{salt \& pepper} e il rumore \textit{Schottky} \cite{arce_2004}.\\
	
	\begin{figure}[ht]
		\centering
		\begin{subfigure}{0.4\linewidth}
			\centering
			\includegraphics[keepaspectratio, width=\linewidth]{images/bs_o2p.png}
			\caption{Immagine originale}
		\end{subfigure}
		\hspace{20pt}
		\begin{subfigure}{0.4\linewidth}
			\centering
			\includegraphics[keepaspectratio, width=\linewidth]{images/bs_o2p_medfilt.png}
			\caption{Immagine con filtro mediana $3\times3$}
		\end{subfigure}
		\caption[Immagine s-SNOM O2P di esemplari di \textit{Bacillus subtilis}]{
			Immagine \acrshort{ssnom} O2P di esemplari di \textit{Bacillus subtilis}}
	\end{figure}
	
	\subsection{Filtro di Wiener}
	
	Il \textbf{filtro di Wiener} è stato ideato per calcolare una stima statistica di un segnale sconosciuto usando un segnale correlato come input, di cui si conosce la componente di rumore, e filtrarlo per produrre la stima. Questo filtro, ideato da \textit{Norbert Wiener} nel 1942\cite{wiener_1942}, ha l'obiettivo di minimizzare l'errore quadratico medio (\acrshort{mse}) tra il processo stocastico stimato e il processo desiderato\cite{oppenheimer_2010}.
	
	Per usare questo filtro, l'immagine e il suo rumore vengono considerati come variabili aleatorie. L'obiettivo è trovare una stima dell'immagine non corrotta $\hat{f}$ tale che l'errore quadratico medio tra di esse sia minimizzato. Questo filtro viene spesso usato per rimuovere una sfocatura di movimento tramite deconvoluzione. 
	
	\begin{equation}
		\text{MSE} = \mathbb{E}\left[\left(f-\hat{f}\right)^2\right]
	\end{equation}
	\\[-10pt]
	Per fare ciò bisogna rispettare dei presupposti: \cite{bergamasco_2016}
	
	\begin{itemize}
		\itemsep0em
		\item Il rumore e l'immagine non devono essere correlati.
		\item Il rumore o l'immagine deve essere centrato, cioè a media nulla.
		\item I livelli di intensità nella stima sono una funzione lineare dei livelli nell'immagine degradata.
	\end{itemize}
	\begin{align}
		\hat{F}(u,v) &= \left[\frac{1}{H(u,v)}\cdot\frac{\left|H(u,v)\right|^2}{\left|H(u,v)\right|^2+S_\eta(u,v)/S_f(u,v)}\right]G(u,v)\\[5pt]
		H(u,v) &= \text{Funzione di degradazione} \nonumber \\
		\left|H(u,v)\right|^2 &= \text{Spettro di potenza di } H \nonumber \\
		S_\eta(u,v) &= \text{Spettro di potenza del rumore} \nonumber \\
		S_f(u,v) &= \text{Spettro di potenza dell'immagine non degradata} \nonumber 
	\end{align}
	
	Se il rumore è bianco con spettro di potenza costante, usando un'approssimazione di $S_f(u,v)$ è possibile sostituire il rapporto con un termine costante.
	
	\begin{equation}
		\hat{F}(u,v)= \left[\frac{1}{H(u,v)} \cdot \frac{\left|H(u,v)\right|^2}{\left|H(u,v)\right|^2+k}\right] G(u,v)
	\end{equation}
	
	\subsection{Filtro bilaterale}
	
	Il \textbf{filtro bilaterale} smussa le immagini preservando i bordi grazie all'utilizzo di una combinazione non lineare di valori nell'intorno del pixel. Questo filtro combina l'intensità dei pixel sia in base alla loro vicinanza geometrica che alla loro similarità fotometrica e preferisce valori vicini tra loro rispetto a valori distanti sia nel dominio che nel codominio\cite{tomasi_1998}.
	
	\begin{equation}
		I_f(x) = \frac{1}{k(x)}\iint_{-\infty}^{\infty}I(x)c(\xi,x)s(f(\xi),f(x))\,d\xi
	\end{equation}
	\\[-10pt]
	La funzione $s(f(\xi),f(x))$ misura la similarità tra il punto $x$ al centro dell'intorno e il punto vicino $\xi$, mentre la funzione $c(\xi,x)$ ne misura la vicinanza geometrica. La funzione $k(x)$ definisce il fattore di normalizzazione.
	
	\begin{equation}
		k(x) = \iint_{-\infty}^{\infty}c(\xi,x)s(f(\xi),f(x))\,d\xi
	\end{equation}
	\\[-10pt]
	Un caso importante del filtro bilaterale è quello in cui sia la funzione di vicinanza che la funzione di similarità sono funzioni gaussiane della distanza euclidea tra i loro argomenti
	\begin{align}
		c(\xi,x) &= \exp\left(-\frac{||\xi-x||^2}{2\sigma^2_d}\right)\\
		s(\xi,x) &= \exp\left(-\frac{||f(\xi)-f(x)||^2}{2\sigma^2_r}\right)
	\end{align}
	
	La diffusione geometrica $\sigma_d$ nel dominio viene scelta in base alla quantità di filtraggio passa-basso desiderata. Allo stesso modo, la diffusione fotometrica $\sigma_r$ nel codominio dell'immagine viene impostata per ottenere la quantità di combinazione desiderata delle intensità dei pixel.
	
	Questo filtro è anche usato per il filtraggio di immagini a colori in quanto consente di combinare opportunamente le tre bande di colore e di misurare le distanze fotometriche tra i pixel nello spazio combinato. Inoltre, questa distanza combinata può essere fatta corrispondere in modo preciso alla dissimilarità percepita dal sistema visivo umano (\acrlong{hvs} --- \acrshort{hvs}) utilizzando la distanza euclidea nello spazio colore CIE-Lab.\cite{wyszecki_2000}
	
	\subsection{Filtro guidato}
	
	Il \textbf{filtro guidato} genera l'immagine di uscita prendendo in considerazione il contenuto spaziale di un'altra immagine, detta di guida, che può essere l'immagine di input stessa o un'altra immagine diversa. Questo filtro ha la proprietà di preservare i bordi come il filtro bilaterale, ma non soffre degli stessi artefatti di inversione del gradiente\cite{durand_2002}.
	\begin{equation}
		f[i] = \sum_{j}I[j]W\{G\}[i,j]
	\end{equation}
	
	L'uscita del filtro è una media pesata in cui il kernel di filtraggio $W\{G\}$ è funzione dell'immagine guida $G$ ed è indipendente dall'immagine da filtrare $I$, mentre $i$ e $j$ sono le coordinate di due pixel. Un esempio di questo kernel è il filtro bilaterale congiunto, che si comporta come un filtro bilaterale quando $G$ e $I$ sono identiche.\cite{petschnigg_2004}
	\begin{equation}
		W^{bf}\{G\}[i,j] = \frac{1}{K} \exp\left(-\frac{|i-j|^2}{\sigma^2_r}\right) \exp\left(-\frac{|G[i]-G[j]|^2}{\sigma_r^2}\right)
	\end{equation} 
	
	Il presupposto fondamentale del filtro guidato è un modello lineare locale tra la guida $G$ e l'uscita del filtro $f$. Assumendo che $f$ sia una trasformazione lineare di $G$ in un intorno $\omega_k$ centrato sul pixel $k$, è sicuro che $f$ presenta un bordo solo se è presente anche in $G$, poiché $\nabla f = a\nabla G$\cite{he_2013}.
	\begin{equation}
		f[i] = a_kG[i]+b_k,\ \forall i\in\omega_k
	\end{equation}
	
	Per determinare i coefficienti lineari, bisogna cercare una soluzione che minimizzi la differenza tra l'immagine in ingresso $I$ e in uscita $f$. La soluzione può essere espressa come una regressione lineare.
	\begin{align}
		a_k &= \frac{\frac{1}{|\omega|}\sum_{i\in\omega_k}G[i]I[i]-\mu_k\bar{I}_k}{\sigma_k^2+\epsilon}\\
		b_k &= \bar{I}_k-a_k\mu_k
	\end{align}
	
	Dopo aver applicato questo modello a tutti gli intorni dell'immagine, può essere calcolato il peso del filtro facendo la media tra tutti gli intorni che includono un certo pixel $i$.
	
	\begin{equation}
		W\{G\}[i,j] = \frac{1}{|\omega|^2} \sum_{(i,j)\in\omega_k} \left[1+\frac{(G[i]-\mu_k)(G[j]-\mu_k)}{\sigma_k^2+\epsilon}\right]
	\end{equation}
	
	\begin{figure}[ht]
		\centering
		\begin{subfigure}{0.32\linewidth}
			\includegraphics[keepaspectratio, width=\linewidth]{images/entero_z.png}
			\caption{Topografia \acrshort{afm}}
		\end{subfigure}
		\begin{subfigure}{0.32\linewidth}
			\includegraphics[keepaspectratio, width=\linewidth]{images/entero_o2a.png}
			\caption{\acrshort{ssnom} O2A}
		\end{subfigure}
		\begin{subfigure}{0.32\linewidth}
			\includegraphics[keepaspectratio, width=\linewidth]{images/entero_guided.png}
			\caption{Immagine filtrata}
		\end{subfigure}
		\caption[Filtro guidato dall'immagine di topografia AFM]{
			Filtro guidato dall'immagine di topografia \acrshort{afm}}
	\end{figure}
	
	\subsection{Filtro a media non locale}
	
	Il \textbf{filtro a media non locale} (NLM), a differenza dei filtri locali come il filtro gaussiano, non determina il valore di un pixel basandosi sulle caratteristiche del suo intorno ma utilizza una media dell'intera immagine, pesata in ogni punto in base alla similarità con il pixel in esame.\cite{buades_2011}.
	
	Data un'immagine rumorosa discreta $v = \left\{v(i) | i\in I\right\}$, il valore stimato dal filtro per il pixel $i$ è calcolato come una media pesata di tutti i pixel dell'immagine, dove la famiglia dei pesi $w(i,j)$ dipende dalla similarità tra le intensità degli intorni $\mathcal{N}_i$ e $\mathcal{N}_j$, rispettivamente dei pixel $i$ e $j$.
	\begin{align}
		\text{NL}[v](i) &= \sum_{j\in I}w(i,j)v(j)\\
		w(i,j) &= \frac{1}{Z(i)}\exp\left(-\frac{\left\lVert v(\mathcal{N}_i) - v(\mathcal{N}_j) \right\lVert^2_{2,a}}{h^2}\right)\\
		Z(i) &= \sum_{j\in I}\exp\left(-\frac{\left\lVert v(\mathcal{N}_i) - v(\mathcal{N}_j) \right\lVert^2_{2,a}}{h^2}\right)
	\end{align}
	
	Questa similarità viene misurata come una funzione decrescente della distanza euclidea pesata, dove $a$ è la deviazione standard del kernel gaussiano. Il parametro $h$ agisce modificando il grado di filtraggio: controlla il decadimento della funzione esponenziale e quindi il decadimento dei pesi in funzione delle distanze euclidee.\cite{buades_2005}\\
	
	\begin{figure}[ht]
		\centering
		\begin{subfigure}{0.32\linewidth}
			\includegraphics[keepaspectratio, width=\linewidth]{images/sa_o2a.png}
			\caption{Originale}
		\end{subfigure}
		\begin{subfigure}{0.32\linewidth}
			\includegraphics[keepaspectratio, width=\linewidth]{images/nlm_0025.png}
			\caption{$h = 0.025$}
		\end{subfigure}
		\begin{subfigure}{0.32\linewidth}
			\includegraphics[keepaspectratio, width=\linewidth]{images/nlm_005.png}
			\caption{$h = 0.05$}
		\end{subfigure}
		\caption{Effetto del parametro $h$ sul filtraggio}
	\end{figure}
	
	Il filtro \acrshort{nlm} non confronta solo le intensità dei singoli pixel, ma anche la configurazione geometrica dell'intero intorno. Questo consente un confronto più robusto rispetto ai filtri locali.
	
	\begin{figure}[ht]
		\centering
		\begin{subfigure}{0.32\linewidth}
			\includegraphics[keepaspectratio, width=\linewidth]{images/nlm_weights_3.png}
			\caption{$3\times3$}
		\end{subfigure}
		\begin{subfigure}{0.32\linewidth}
			\includegraphics[keepaspectratio, width=\linewidth]{images/nlm_weights_5.png}
			\caption{$5\times5$}
		\end{subfigure}
		\begin{subfigure}{0.32\linewidth}
			\begin{tikzpicture}
				\begin{scope}[
					node distance = 11mm,
					inner sep = 0pt,spy using outlines={rectangle, red, magnification=6}]
					\node (n0)  {\includegraphics[keepaspectratio, width=\textwidth]{images/nlm_weights_7.png}};
					
					\spy [white,size=1cm] on (n0.center) in node[below right=of n0.north west];
				\end{scope}
				\draw[dashed,white] (tikzspyonnode.north east) -- (tikzspyinnode.north east);
				\draw[dashed,white] (tikzspyonnode.south west) -- (tikzspyinnode.south west);
			\end{tikzpicture}
			\caption{$7\times7$}
		\end{subfigure}
		\caption[Distribuzione dei pesi rispetto al	pixel centrale dell'immagine]{\begin{tabular}[t]{@{}l@{}}
				Effetto della grandezza dell'intorno sulla distribuzione dei pesi rispetto al\\
				pixel centrale dell'immagine
		\end{tabular}}
	\end{figure}
	
	\subsection{Filtro BM3D}
	
	Il \textbf{filtro \acrshort{bm3d}} (\acrlong{bm3d}) è un'estensione del filtro a media non locale che fa uso di una migliore rappresentazione sparsa nel dominio di trasformazione\cite{manjon_2008}.
	
	Il primo passo consiste nel suddividere l'immagine in blocchi (patch) di dimensioni fisse e poi raggruppare insieme i blocchi simili da un algoritmo di block-matching, formando un tensore tridimensionale (gruppo) in cui ogni strato del volume è un blocco simile. Questi blocchi non sono necessariamente disgiunti ma possono anche sovrapporsi parzialmente.
	
	Una volta raccolti i blocchi simili, è possibile applicare delle funzioni di trasformazione (come DCT o wavelet) per rappresentare questi gruppi in modo più compatto. Nello spazio trasformato, il rumore appare come elementi sparsi, mentre i dati reali (cioè i dettagli coerenti tra i blocchi) si concentrano in poche componenti. Si può quindi applicare un filtro sulla trasformata, come un filtro di Wiener, per ridurre il rumore, eliminando i coefficienti minori.\cite{maggioni_2013}
	
	Dopo il filtraggio, il tensore viene ricostruito tramite la trasformata inversa e i blocchi ripuliti vengono reinseriti nell'immagine. Dato che un pixel può comparire in più blocchi, l'immagine finale si ottiene come media pesata di tutti i contributi.\cite{dabov_2007}
	
\end{document}