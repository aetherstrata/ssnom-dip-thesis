\documentclass[../main.tex]{subfiles}
\begin{document}

\chapter{Conclusioni}

Lo studio di microrganismi tramite tecniche di microscopia a super-risoluzione è fondamentale per comprendere la loro struttura e i loro comportamenti. I batteri ESKAPE\cite{eskape} sono un serio rischio per la salute\cite{rice_2008} e una minaccia sempre crescente nel futuro.\cite{pipito_2025} Studiare questi batteri con le tecniche descritte in questa tesi può portare a una migliore comprensione dei fattori di virulenza e sviluppare nuovi farmaci che superano le loro resistenze.

Effettuare questi studi con tecniche di microscopia come \acrshort{afm} e \acrshort{snom}, al posto di tecniche come la microscopia elettronica, ha il beneficio di non richiedere trattamenti speciali per i campioni, che invece potrebbero danneggiare o distruggere il campione. La microscopia \acrshort{afm} fornisce un profilo tridimensionale del campione, da cui si possono anche ottenere le sue proprietà viscoelastiche, mentre la microscopia \acrshort{snom} fornisce informazioni sulle sue proprietà ottiche, come l'indice di rifrazione.

Queste due tecniche di microscopia funzionano senza aver bisogno di ambienti particolari, come una camera a vuoto o criogenica, e riescono a osservare organismi viventi senza danneggiarli. Inoltre, queste tecniche possono essere combinate nello stesso microscopio per ottenere più informazioni correlate della stessa scansione.

La microscopia \acrshort{ssnom} ha suscitato interesse negli ultimi anni, ma il suo impiego nella \gls{microbiologia} è rimasto limitato, soprattutto per le difficoltà legate all'interpretazione dei dati. Lo stesso microscopio può fornire immagini \acrshort{afm} e \acrshort{snom} dello stesso campione e questo recenti studi hanno utilizzato un sistema di microscopia \acrshort{ssnom} insieme ad altri sistemi come la microscopia \acrshort{clsm}.\cite{stanciu_2017}

\begin{figure}[h]
	\centering
	\includegraphics[keepaspectratio, height=\linewidth]{images/multimodal_system.jpg}
	\caption[Sistema multimodale per l'acquisizione di immagini correlative]{
		Sistema multimodale per l'acquisizione di immagini correlative \cite{stanciu_2017}}
	\label{fig:multimodal_system}
\end{figure}


Uno dei problemi nell'elaborazione di immagini \acrshort{afm} e \acrshort{snom} è la quantità di rumore. Uno dei parametri che maggiormente influenza IL rumore è il tempo di integrazione per pixel della scansione: più è alto questo parametro, più il rumore sarà attenuato. Tuttavia, il tempo di integrazione alto aumenta il rischio di dislocare il campione e aumenta il tempo di scansione. Nelle le immagini del dataset  SSNOMBACTER è stato usato un tempo di integrazione di $3$ ms, che equivale a un tempo di scansione di quasi $5$ minuti per singola immagine.

Usare delle tecniche di riduzione del rumore permette di aumentare i dettagli mantenendo un tempo di integrazione basso e riducendo il tempo richiesto per l'acquisizione. Dalle valutazioni esposte in questa tesi si evince come, per immagini in cui il rumore di tipo impulsivo sia elevato è consigliato usare un filtro mediana al costo di una leggera sfocatura dei dettagli, mentre è consigliato usare un filtro a media non locale in immagini con poco rumore impulsivo per rimuovere in modo efficace il rumore gaussiano mantenendo intatti i dettagli del campione.

\section{Sviluppi futuri}

La scarsa disponibilità di dati non permette l'uso di tecniche di deep learning per classificare la qualità delle immagini o per ridurre il rumore in modo affidabile senza incorporare dati esterni, come dei \textit{rank} o punteggi soggettivi come il \acrshort{mos} (\acrlong{mos}), visto che per poter usare queste tecniche più moderne è richiesta una grande quantità di dati per l'apprendimento.\cite{lecun_2015,litjens_2017,bosse_2018} Una soluzione sarebbe quella di acquisire più scansioni ma, data la grande quantità di tempo richiesta per effettuarle, si può prendere in considerazione anche l'uso di AI generativa, come un modello di diffusione, per estendere il dataset con immagini sintetiche partendo da quelle presenti.\cite{ho_2020}

Questo studio può essere esteso ad altri tipi di filtri basati sulla mediana per valutare la loro efficacia nel rimuovere il rumore impulsivo su questo tipo di immagini. Di seguito è presentata una lista di alcuni possibili filtri:

\begin{itemize}
	\itemsep 0em
	\item \acrfull{mf}\cite{gallagher_1981} --- il filtro usato in questa tesi
	\item \acrfull{wmf}\cite{zhang_2009}
	\item \acrfull{cwmf}\cite{ko_1991}
	\item \acrfull{rwmf}\cite{kumar_2007}
	\item \acrfull{amf}\cite{chen_2001}
	\item \acrfull{asmf}\cite{khryashchev_2005}
	\item \acrfull{psmf}\cite{wang_1999}
	\item \acrfull{tsmf}\cite{chen_1999}
\end{itemize}

Le immagini ottenute dall'applicazione delle tecniche di riduzione del rumore possono essere usate in applicazioni di visione artificiale, per esempio per effettuare il conteggio delle cellule, lo studio delle loro caratteristiche morfologiche, come la rugosità della parete, la sua conformazione e come si dispongono gruppi di cellule. Un altro campo di interesse molto importante è la classificazione delle specie partendo dalle immagini o lo studio della mobilità dei batteri con flagelli e il loro riconoscimento.

Le tecniche di riduzione del rumore possono anche essere valutate secondo delle statistiche proprie di queste applicazioni finali. In questa tesi viene portato come esempio un processo di \textit{instance segmentation} automatizzato\cite{hafiz_2020}. Per prima cosa sono state create manualmente delle regioni di interesse corrispondenti ai batteri sulle immagini, in questo caso tutte della stessa classe. Poi è stato fatto il fine-tuning di un modello \acrshort{yolo} (\acrlong{yolo}) sulla maggior parte delle immagini mentre il restante $20\%$ è stato usato per la validazione dei risultati. Per fare ciò è stato usato il modello YOLO11 di Ultralytics\cite{ultralytics_2023}, licenziato secondo la licenza AGPL-3.0, su un sistema con un processore Ryzen 7 3700x, 32GB di RAM e una scheda video NVIDIA RTX 3070Ti. Il modello ha dato buoni risultati, con il filtro mediana che fornisce risultati leggermente migliori rispetto al filtro \acrshort{nlm}.\medskip

\begin{table}[ht]
	\centering
	\begin{tabular}{r|cccc}
	\textbf{Risultati} & \textbf{Precisione} & \textbf{Recupero} & \textbf{mAP50} & \textbf{mAP50-95} \\
	\hline\hline
	Mediana & 77.3\% & 83.84\% & 82.76\% & 59.17\% \\
	NLM & 76.08\% & 80.32\% & 81.16\% & 61.11\%
	\end{tabular}
	\caption{Statistiche della segmentazione}
\end{table}

\begin{figure}[ht]
	\includegraphics[keepaspectratio,width=\linewidth]{images/seg_val.png}
	\caption{Alcuni risultati della segmentazione}
\end{figure}

\end{document} 
