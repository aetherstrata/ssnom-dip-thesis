\documentclass[../main.tex]{subfiles}
\begin{document}
\chapter*{Introduzione}
\addcontentsline{toc}{chapter}{Introduzione}

Sin dal 17\textdegree\ secolo, in cui Antonie van Leeuwenhoek pose le basi della Microbiologia\cite{lane_2015, dobell_1923, corliss_1975}, i microscopi ottici sono stati di vitale importanza nello sviluppo della nostra comprensione degli organismi microscopici. In tempi recenti, il raggiungimento del limite teorico della risoluzione spaziale dei microscopi ottici convenzionali, proporzionale alla lunghezza d'onda della luce, ha favorito lo sviluppo di nuove tecniche di microscopia a super-risoluzione.\cite{ohtsu_2020}

In questa tesi sono state prese in esame delle immagini generate da un microscopio \acrshort{ssnom}, che combina le tecniche di microscopia \acrshort{afm} e \acrshort{snom}. Queste tecniche hanno una risoluzione laterale di pochi nanometri e registrano diverse proprietà del campione esaminato. 

Le immagini \acrshort{afm} sono generate facendo scorrere un cantilever con una punta di circa $10$ nm sulla superficie del campione. Durante la scansione la punta viene inclinata verso l'alto e il basso dalle forze di interazione con il campione, e queste inclinazioni vengono poi misurate da un raggio laser che viene riflesso dal cantilever su un fotodiodo a quadranti. Il segnale misurato da questo fotodiodo viene convertito in una variazione di altezza della punta. In questo caso è stata impiegata la modalità di funzionamento a contatto intermittente, in cui il cantilever è messo in vibrazione alla sua frequenza di risonanza e l'ampiezza di oscillazione è regolata in modo da toccare appena il campione. Questo è importante quando bisogna lavorare con campioni che possono dislocarsi o danneggiarsi facilmente a causa delle forze impresse dalla punta su essi.

Le immagini \acrshort{snom} sono generate sfruttando alcune proprietà dei campi evanescenti per misurare le proprietà ottiche del campione con una risoluzione molto maggiore rispetto alle tecniche di microscopia ottica tradizionale. Poiché le onde evanescenti sono confinate a una regione prossima alla superficie del mezzo che le origina, è necessario portare il rilevatore entro l'intervallo della lunghezza d'onda della radiazione utilizzata. In questo caso è stato impiegato un laser con una lunghezza d'onda di $1550$ nm che concentra il fascio luminoso nell'area tra la punta e il campione, generando un campo evanescente.

Le immagini in esame in questa tesi sono prese dal dataset di SSNOMBACTER, un articolo scientifico pubblicato  nel 2020 su GigaScience.\cite{ssnombacter} In questo studio sono state generate oltre 4000 immagini \acrshort{afm} e \acrshort{snom} di 15 specie di batteri diversi, tra cui anche le specie del gruppo ESKAPE\cite{eskape}. Queste specie di batteri sono di particolare interesse perché sono resistenti agli antibiotici o sono state identificate come patogeni opportunisti e rimangono tuttora un grave problema per la salute pubblica, anche nei paesi sviluppati.\cite{rice_2008} Infatti, la resistenza agli antibiotici è una crescente minaccia per la salute globale, che si prevede sarà la causa di oltre 40 milioni di morti entro il 2050.\cite{pipito_2025}

Studiare questi batteri con tecniche di microscopia a super-risoluzione è di fondamentale importanza per comprendere le interazioni a livello molecolare tra i batteri e le cellule ospite, per studiare i fattori di virulenza e potenzialmente sviluppare dei farmaci più efficaci.

Nel lavoro svolto in questa tesi sono state prese in esame le immagini di topografia \acrshort{afm} e le ampiezze della $2^a$ e $3^a$ armonica \acrshort{ssnom} per valutare l'efficacia di diverse tecniche di riduzione del rumore su questo tipo di immagini. Uno dei fattori che influenza la quantità di rumore in queste immagini è il tempo di integrazione per pixel del microscopio: impostare un periodo di tempo più alto permette di avere immagini con meno rumore a discapito di un tempo di scansione più lungo o il rischio di dislocare il campione. Usare delle tecniche di riduzione del rumore su immagini ottenute con un tempo di integrazione basso, come nel caso delle immagini del dataset SSNOMBACTER, permette di aumentare la qualità delle immagini senza dover aumentare il tempo richiesto per l'acquisizione delle immagini.\cite{baiz_2025} 

Le tecniche prese in considerazione in questa tesi sono il filtro gaussiano, il filtro rettangolare, il filtro mediana, il filtro di Wiener, il filtro bilaterale, il filtro guidato, il filtro a media non locale e il filtro \acrshort{bm3d}. Per valutare l'efficacia di questi filtri sono stati usati diversi metodi di valutazione. Non essendo disponibili delle immagini di riferimento o valutazioni soggettive di esperti, sono stati usati dei metodi di valutazione senza riferimento (\acrshort{nriqa}). Oltre a questi, è stata fatta un'analisi comparativa tra le caratteristiche delle immagini di ampiezza \acrshort{snom} e quelle della corrispondente immagine di topografia \acrshort{afm}, usando misure come la mutua informazione o la scala \acrshort{fsim}.

Dopo aver testato questi filtri su tutte le $93$ scansioni e preso la media delle valutazioni, sono analizzati i risultati ottenuti da cui si evince come, complessivamente, le tecniche migliori siano il filtro mediana o il filtro \acrshort{nlm}. Questo dipende da che tipo di rumore sia più tollerato dalle tecniche usate per elaborare le immagini. Il filtro mediana rimuove gran parte del rumore impulsivo e una parte del rumore bianco, ma trasforma le variazioni graduali in gradini. Invece, il filtro \acrshort{nlm} riduce gran parte del rumore bianco mantenendo comunque inalterate le caratteristiche del campione, ma lascia inalterato il rumore impulsivo sui campioni, quindi è importante scegliere il metodo giusto in base alle esigenze.\\

% possibile paragrafo su futuri sviluppo

\noindent Questa tesi è strutturata in 5 capitoli. Nel primo capitolo vengono descritte le principali tecniche di microscopia a super-risoluzione, con particolare interesse verso le tecniche di microscopia \acrshort{afm} e \acrshort{snom}. Nel secondo capitolo viene descritta la modalità di acquisizione delle immagini presenti in SSNOMBACTER\cite{ssnombacter}, si presenta una panoramica sui batteri ESKAPE\cite{eskape} e sulle motivazioni dell'uso di queste tecniche per studiarli. Nel terzo capitolo vengono presentate le tecniche di riduzione del rumore prese in esame in questa tesi. Nel quarto capitolo vengono esposti i metodi di valutazione usati per studiare l'efficacia dei filtri e discussi i risultati ottenuti. Nel quinto capitolo è esposto un riepilogo dei risultati raggiunti e possibili usi per sviluppi futuri.

\end{document}
  