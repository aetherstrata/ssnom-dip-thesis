\documentclass[../main.tex]{subfiles}
\begin{document}
\chapter*{Introduzione}
\addcontentsline{toc}{chapter}{Introduzione}
Sin dal 17\textdegree\ secolo, in cui Antonie van Leeuwenhoek pose le basi della Microbiologia\cite{lane_2015, dobell_1923, corliss_1975}, i microscopi ottici sono stati di vitale importanza nello sviluppo della nostra comprensione degli organismi microscopici.
In tempi recenti, il raggiungimento del limite teorico della risoluzione spaziale dei microscopi ottici convenzionali, proporzionale alla lunghezza d'onda della luce, ha favorito lo sviluppo di nuovi dispositivi ottici che operano nel campo prossimo, come i microscopi \gls{snom}\cite{ohtsu_2020}.

Nel 2020 è stato pubbicato sulla rivista GigaScience un set di oltre 4000 immagini di 15 specie di batteri diverse acquisite con un microscopio NeaSNOM\cite{ssnombacter}. I batteri rimangono tuttora un grave problema per la salute pubblica, anche nei paesi sviluppati, per questo tra le specie esaminate in questo articolo ci sono anche batteri del gruppo ESKAPE, che sono resistenti agli antibiotici o sono stati identificati come patogeni opportunisti.\cite{eskape}

In questa tesi vengono analizzate varie tecniche di elaborazione per estrarre informazioni utili, anche in modo automatico, da queste immagini, partendo dalle diverse modalità di acquisizione e quali proprietà del campione possono essere registrate.
Dopo aver discusso le procedure di elaborazione utilizzate, e come variano in base al tipo di immagine presa in considerazione, vengono tratte le conclusioni su quali caratteristiche possono essere estratte e quali tipi di immagini acquisite sono più utili al lavoro.
\end{document}
